% -*- coding: UTF-8 -*-
\documentclass[twoside,nofonts,fancyhdr,openany,UTF8,fleqn]{ctexart} % 设置文章类型
\usepackage{CJK}
\usepackage{amsmath}
\usepackage{amssymb}

\usepackage[colorlinks,linkcolor=red]{hyperref}  % 添加链接的宏包


% 设置字体
\setCJKmainfont[AutoFakeBold=true]{Adobe Song Std} 
\setCJKsansfont{Adobe Heiti Std}
\setCJKmonofont{Adobe FangSong Std}

\title{CS24N Assignment1}    	% 标题
\author{Jerry.Shi}		% 作者

\begin{document}	% 正文开始

%\maketitle			% 制作封面

%\tableofcontents % 加入目录,包括页码

\section{Softmax}
\subsection{a}
证明softmax函数对于任意的向量\textbf{\emph{x}} 和任意常数\emph{c}存在以下关系,
\begin{equation}
softmax(\textbf{\emph{x}}) = softmax(\textbf{\emph{x}} + \emph{c})
\end{equation}
上式表示对向量的每个维度同时加上同一个常数softmax的结果不变,softmax公式如下,
\begin{equation}
softmax(\textbf{\emph{x}})_i = \frac{exp(\emph{x}_i)}{\sum_{j}exp(\emph{x}_j)}
\end{equation}
一般情况下为了保持数值稳定性会选择$\textbf{c}=-\textbf{max}(\textbf{\emph{x}}_i)$, 让每个维度都减去最大的那个元素,进行稳定性处理。

Solution:
根据softmax的公式得到,
% 推导过程

\begin{align}
softmax(\textbf{\emph{x}+c})  & =  \frac{exp(\textbf{x}_i + \textbf{c})}{\sum_{j}exp(\emph{x}_j+\textbf{c})} \\
								  & =  \frac{exp(\textbf{x}_i)exp(\textbf{c})}{\sum_{j}exp(\emph{x}_j)exp(\textbf{c})}	\\
								  & = \frac{exp(\textbf{x}_i)}{\sum_{j}(exp(\emph{x}_))}	\\
								  & = softmax(\textbf{\emph{x}}) 
\end{align} % 这里不要有空行否则会出错

\subsection{b}
给定一个\textbf{N}x\textbf{D}的矩阵,使用题目a中的特性求解矩阵中每一行的softmax结果,在文件q1softmax.py中给出实现和测试。



\section{Neural Network Basics}
\subsection{a}
推导sigmod函数的梯度,并将其表示成其他函数的形式,sigmod函数如下,
\begin{equation}
 \sigma(\textbf{x}) = \frac{1}{1+exp(-\textbf{x})} 
\end{equation}
Solutions:
\begin{align}
\sigma^\prime(\textbf{x})  & = \frac{exp(-\textbf{x})}{(1+exp(-\textbf{x}))^2} \\
		& = \frac{1}{1+exp(-\textbf{x})}\frac{exp(x)}{1+exp(-\textbf{x})} \\
	& = \frac{1}{1+exp(-\textbf{x})}(1- \frac{1}{1+exp(-\textbf{x})})	\\
	& = \sigma(\textbf{x})(1-\sigma(x))
\end{align}

\subsection{b} 
推导激活函数为softmax,loss 为 cross entropy的梯度,输入向量$\theta$,预测结果$\hat{\textbf{\emph{y}}}=softmax(\theta)$,交叉熵cross entropy公式为,
\begin{equation}
CE(\textbf{\emph{y}}, \hat{\textbf{\emph{y}}}) = - \sum_{i}y_ilog(\hat{y}_i)
\end{equation}

其中向量\textbf{\emph{y}}为one-hot编码,$\hat{\textbf{\emph{y}}}$是对所有class的预测概率向量。

Solution:
对$\theta$求导







\end{document}
